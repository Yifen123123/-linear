\documentclass
[answers]
{exam}
\usepackage[english]{babel}
\usepackage[utf8x]{inputenc}
\usepackage[T1]{fontenc}
\usepackage[a4paper,margin = 2cm]{geometry} 
\usepackage{amsmath,amsthm,amssymb}
\usepackage{graphicx}
\usepackage{tasks}
\usepackage{paralist}
\usepackage{mathrsfs}
\newcommand{\N}{\mathbb{N}}
\newcommand{\Z}{\mathbb{Z}}
\everymath{\displaystyle}
\usepackage{xeCJK}   % Chinese input settings
\setCJKmainfont{標楷體} % Windows使用者請使用這行

\begin{document}
\begin{questions}
\question State the $dimension\;theorem.$
\begin{solution}
\begin{flalign*}
  <1>&V,W : V.S\\
  &T:V \rightarrow W\ be\ linear\\
  &V \;is\; a\; finite-dimensional\\
  &dim(V)=nullity(T)+rank(T)&
\end{flalign*}
\end{solution}


\question 
Show that if $T:V \rightarrow V$ is a linear transformation, and $V$ is finite dimensional vector space, then $T$ is $1-1$ if and only if $T$ is onto.
\begin{solution}
\begin{flalign*}
  <1>&(\rightarrow) \because one\ to\ one\\
  &\therefore nullity=0, By\ dimension\ theroem\\
  &dim(V)=rank(T)+nullity(T)\\
  &dim(V)=rank(T)+0=rank(T)\\
  &Then\ T: V \rightarrow V\\
  &dim(V)=dim(V)\\
  &\therefore T\ is\ onto\\
  &(\leftarrow)\because T\ is\ onto\\ 
  &Then\ T: V \rightarrow V\\
  &dim(V)=dim(V)\\
  &Hence\ T\ is\ one\ to\ one&
\end{flalign*}  
\end{solution}

\question
For Exercises 2 through 6, prove that T is a linear transformation, and find bases for both N(T) and R(T). Then compute the nullity and rank of T, and verity the dimension theorem. Finally, use the appropriate theorems in this section to determine whether T is one-to-one or onto.

4. $T: M_{2\times2}(F)\rightarrow M_{2\times2}(F)$ defined by 
\[
T\begin{pmatrix}
  a_{11} & a_{12} & a_{13} \\
  a_{21} & a_{22} & a_{23} \\
\end{pmatrix} 
= 
\begin{pmatrix}
  2a_{11}-a_{12} & a_{13}+2a_{12} \\
  0 & 0 \\
\end{pmatrix}
\]

\begin{solution}
\begin{flalign*}
  <1>&proof:\\
  &\text{1. T is linear}\\
  &\text{2. bases for N(T), R(T)}\\
  &\text{3. nullity(T), rank(T)}\\
  &\text{4. dimension theorem}\\
  &\text{5. T is one-to-one or onto}\\
  &1.\\
  &\text{(1) 
  $T(0)=T\begin{pmatrix}
  0 & 0 & 0 \\
  0 & 0 & 0 \\
  \end{pmatrix}
  =\begin{pmatrix}
  0 & 0 \\
  0 & 0 \\
  \end{pmatrix}$}\\
  &\text{(2) Given $x=\begin{pmatrix}
  a_{11} & a_{12} & a_{13} \\
  a_{21} & a_{22} & a_{23} \\
  \end{pmatrix}$, $y=\begin{pmatrix}
  b_{11} & b_{12} & b_{13} \\
  b_{21} & b_{22} & b_{23} \\
  \end{pmatrix}$}\\
  &\text{$T(cx+y)=T
  \left(
  \begin{pmatrix}
  ca_{11} & ca_{12} & ca_{13} \\
  ca_{21} & ca_{22} & ca_{23} \\
  \end{pmatrix}+
  \begin{pmatrix}
  b_{11} & b_{12} & b_{13} \\
  b_{21} & b_{22} & b_{23} \\
  \end{pmatrix}
  \right)$}\\
  &\text{$=T\begin{pmatrix}
  ca_{11}+b_{11} & ca_{12}+b_{12} & ca_{13}+b_{13} \\
  ca_{21}+b_{21} & ca_{22}+b_{22} & ca_{23}+b_{23} \\
  \end{pmatrix}$}\\
  &\text{$=\begin{pmatrix}
  2(ca_{11}+b_{11})-(ca_{12}+b_{12}) & ca_{13}+b_{13})+2(ca_{12}+b_{12}) \\
  0 & 0 \\
  \end{pmatrix}$}\\
  &\text{$=\left(
  \begin{pmatrix}
  c(2a_{11}-a_{12}) & c(a_{13}+2a_{12}) \\
  0 & 0 \\
  \end{pmatrix}+
  \begin{pmatrix}
  2b_{11}-b_{12}) & b_{13}+2b_{12} \\
  0 & 0 \\
  \end{pmatrix}
  \right)$}\\
  &\text{$=cT(x)+T(y)$}\\
  &2.\\
  &\text{$(1) T(x)=0
  \Rightarrow
  \begin{cases}
	2a_{11} - a_{12}=0 \\
	a_{13} + 2a_{12}=0\\  
  \end{cases}
  \Rightarrow
  \begin{cases}
	2a_{11}=a_{12} \\
	a_{13}=-2a_{12}=-4a_{11}\\  
  \end{cases}$}\\
  &\text{$\Rightarrow N(T)=
  \left\{
  \begin{pmatrix}
  a_{11} & 2a_{11} & -4a_{11}\\
  a_{21} & a_{22} & a_{23} \\
  \end{pmatrix} \mid a\in F \right\}$}\\
  &\text{$\Rightarrow$ basis for N(T) :$
  \left\{
  \begin{pmatrix}
  1 & 2 & -4\\
  0 & 0 & 0 \\
  \end{pmatrix},
  \begin{pmatrix}
  0 & 0 & 0\\
  1 & 0 & 0 \\
  \end{pmatrix},
  \begin{pmatrix}
  0 & 0 & 0\\
  0 & 1 & 0 \\
  \end{pmatrix}
  \begin{pmatrix}
  0 & 0 & 0\\
  0 & 0 & 1 \\
  \end{pmatrix}
  \right\}$}\\
  &\text{(2) basis for R(T) :
  $\left\{
  \begin{pmatrix}
  1 & 0 \\
  0 & 0 \\
  \end{pmatrix},
  \begin{pmatrix}
  0 & 1 \\
  0 & 0 \\
  \end{pmatrix}
  \right\}$}\\
  &3.\\
  &\text{$nullity(T)=4$}\\
  &\text{$rank(T)=2$}\\
  &4.\\
  &\text{$dimension\ theroem$}\\
  &\text{dim(V)=rank(T)+nullity(T)}\\
  &\text{$\therefore dim(M_{2\times3)}=nullity(T)+rank(T)$}\\
  &\text{$\therefore 6=4+2$}\\
  &5.\\
  &\text{T is not 1-1 $\because 
  N(T)=\left\{\begin{pmatrix}
  a_{11} & 2a_{11} & -4a_{11}\\
  a_{21} & a_{22} & a_{23} \\
  \end{pmatrix} \mid a\in F \right\}$}\\
  &\text{T is not onto $\because
  R(T)= \left\{\begin{pmatrix}
  x & y\\
  0 & 0\\
  \end{pmatrix}\mid  x,y\in F \right\}$}&
\end{flalign*}  
\end{solution}

\question
Let $V$ be a vector space, and let $T:V\rightarrow V$ be linear. Prove that $T^2 = T_0$ if and only if $R(T)\subseteq N(T)$

\begin{solution}
\begin{flalign*}
 <1>&\text{$(\Rightarrow)$}\\
 &\text{Suppose $T^2=T_0$}\\
 &\text{$\forall  u\in R(T) ,\exists v\in V \ni $}\\
 &\text{$Tv=u, 0=T^2 v=T(T(v))=T(u)$}\\
 &\text{$\therefore u \in N(T)$}\\
 &\text{$Thus\ R(T) \subseteq N(T)$}\\
 &\text{$(\Leftarrow)$}\\
 &\text{$Suppose\ R(T) \subseteq N(T)$}\\
 &\text{$\forall\ v \in V, T(v) \in R(T)$}\\
 &\text{$\Rightarrow T(v) \in N(T)$}\\
 &\text{$Thus\ 0=T(T(v))=T^2 v$}\\
 &\text{$\therefore T^2 = T_0$}&
\end{flalign*}
\end{solution}
\end{questions}


\end{document}

